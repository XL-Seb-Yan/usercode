
CMS Tier2's provide a large amount of resources in a fully distributed way,
spanning various
Grid organizations and middleware stacks
without any central place playing a reference role.
Each computing center has
different configuration, user community, policy, history and
constraints.
It was clear since the early days of CMS Analysis that
good monitoring tools are critical to the success of such
a highly distributed project, both to keep an overall
view of the usage and use patterns and to make it possible
to detect and address problems in an effective way,
using quantitative information to judge where to put
more effort and to evaluate the effectivness of solutions.
Good monitoring has allowed CMS to evolve tool development
and operational structure from vision and anedoctal driven
to a fact based approach.

Since the beginning a few main point drove the development of
those monitoring tools:
\begin{itemize}
\item no reliance on local site monitoring. It is not possible to ask
 all Tier2's to install some kind of CMS'owm ad hoc software, only the
 minimal grid service needed for grid interoperability can be expected.
\item a single entry point to the high level view of the usage from which
 it would be possible to drill down to detailed information at the single
  job level
\item  monitoring is not accounting: keep the system lean and
 flexible and do not worry if a small
 percentage of jobs is not properly reported
\item diagnose failures of users workflows so to
  take suggest corrective actions to developers and administrators, make
  enough proper information available so that plans and action are set
  based on quantitative facts and effectiveness of solutions can be metered
\item detect overall usage patterns to guide management into making
 choices and plans about how and were to steer user activities and
 how to plan for the future
\item do not try to define a priori the relevant metrics, usage, tools and
 hardware will always be changing and we expect the need to fight the
  problem of the month, different each month
\end{itemize}

That requires the capability to collect for most (ideally all) of the jobs
information like: execution and submission site, user, kind of activity,
accessed data, amount of I/O and CPU, efficiency of resource usage,
failure details (if any) and more and store it in some
place where data can be queried in various ways according to need.

Therefore job monitoring was built around the idea of instrumented
application: CMS jobs and tools send messages with a few relevant
parameters to a central collector, independently and in parallel to
whichever grid or monitoring system may exists, thus giving us
access to experiment specific information and giving independency
from execution sites. The price to pay is that only jobs that use the
instrumented submission framework can be monitored in this way.
This price turns out to be very small in CMS case, since almost
all user analysis jobs are submitted using the CRAB tool, which
inserts on user's behalf the needed reporting hooks.

The implementation of this idea comprises an optimised central
database running on Oraclo server, a set of information collectors
that feed data into it from various sources, a simple messaging
system to send information to the collector from the
CRAB tools running at submission location and job execution hosts,
and a few web pages providing access with different levels
of detail, aggregation and flexibility, customized to
typical use cases. This set of tools is called ''the CMS Dashboard''.

The Dashboard main collector is the one receiving messages from
CRAB user interface and CRAB wrapper used in running jobs, for thi
we used the already existing MonAlisa messaging system, which was
already proven to the needed scale. Other collectors add
information from gLite WMS and gLite LB hosts, adding in this way
information about the status of jobs while processed by the
middleware or local batch systems and allowing some limited
tracking of jobs submitted outside the CRAB tool.

The CMS dashboard collects information for both user jobs
(Analysis) and centrally controlled jobs (Production), providing
a comprehensive and uniform view of the CMS usage of computing
resources.

Three web interfaces are used to monitor and metere distributed analysis,
since those are different views of the same underlying database,
it is possible to cross link and navigate from one to another
providing both extreme flexibility and fast access to desired
informations

\subsection{History View}
The aim of this view is to present time history of relevant
metrics to get a view of overall patterns and trends.
\figure{put a screenshot}
The list of viewable metrics is thus predefined, and
the interface uses aggregated tables in the DB to provide
efficient access to old information with (of course) a limited
amount of detail. Metrics can be viewed up to one year in the past
with granularity varying from fraction of a hour to a few days
according to the chosen time period. Data can be viewed divided
according e.g. to the used site(s), job type or job completion status.


\subsection{Task Monitoring for Analysis user}
Users usually submit many jobs in a single workflow, this
is called a Task in CRAB. Tasks usually have from a few hundred
to a thousand jobs each. A user will oftern submit one crab task
for each dataset used in a particular analysis, up to a few
tasks in one day. Task Monitoring for Analysis user was
developer as a user centric interface where the user is
initially presented the list of tasks he submitted in lasts
three days, both in tabular and graphical way.
\figure{a screenshot}
The user can then expand one selected task to get
a graphical overall summary of execution/completion progress
and access details of each job.1
\figure{another screenshot}

\subsection{Interactive Interface}
This view was the first to be developed, based on
vision more then experience with the tool, therefore
emphasis was put on flexibility. It is a job-centric view
where the entry point is the number of jobs submitted or
terminated in a chosen time period.
\figure{a screenshot}
The interactive interface
allows then to drill down expanding the set of jobs by
various relevant properties (execution site, grid gateway,
submitter user, completions status, grid workload management host,
activity type, used dataset etc.), until all details stored in the DataBase
about a choosen (set of) job(s) can be accessed.
The interface reports success/failure values and fractions
according to grid/site/application problem and information
on used wall and cpu time of jobs.
The DataBase ahs been setup so to allow fast access for this
detailed information for the last two weeks, while queries
become more and more slow as time range expands. The idea is that
details are only needed to analyse current issues and for
long time view of trends the History View should be used.

